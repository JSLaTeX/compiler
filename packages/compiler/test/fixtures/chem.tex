\documentclass{article}
\usepackage{float}

<?
const rawData = [
  [20, 24.4, 72, 75.358, -0.0417, -0.9918],
  [20, 24.2, 132, 76.1, -0.0521, -0.9591],
  [20, 24.2, 60, 74.9, -0.086299, -0.99338],
  [20, 24.8, 72, 38.948, -0.0052021, -0.95144],
  [20, 24.6, 96, 75.883, -0.042188, -0.99564],
  [20, 23.1, 14, 35.707, +0.0060035, 0.952365476],
  [20, 23.8, 15, 74.526, -0.059536, -0.8973],
  [20, 24.6, 96, 61.881, -0.02918, -0.99694],
];

const keys = ['volume', 'initialTemp', 'zincAdditionTime', 'coolingLineB', 'coolingLineM', 'rValue'];

// The Ramda library (https://ramdajs.com/) is present in the global scope as `R`
const data = rawData.map((row) => Object.fromEntries(R.zip(keys, row)));

function getRowEquation(rowIndex) {
  const row = data[rowIndex];
  const sign = row.coolingLineM > 0 ? '+' : '-';
  const equation = `${row.coolingLineM.toFixed(3)} ${sign} ${Math.abs(row.coolingLineM).toFixed(3)}$x$`;
  return equation;
}
?>

\begin{document}
	\begin{table}[H]
		\begin{tabularx}{\linewidth}{|
				>{\RaggedRight}X|
				>{\RaggedRight}X|
				>{\RaggedRight}X|
				>{\RaggedRight}X|
				>{\RaggedRight}X|
				>{\RaggedRight}X|
			}
			\hline
			Trial \#
			& Volume of \ce{CuSO4} /\ml
			& Baseline temperature /\celsius
			& Time of \ce{Zn} addition /\second
			& Equation of cooling line
			& R value
			\\\hline
			<? for (const [rowIndex, row] of data.entries()) { ?>
				<?= rowIndex + 1 ?>
				& <?= row.volume ?>
				& <?= row.initialTemp ?>
				& <?= row.zincAdditionTime ?>
				& <?= getRowEquation(rowIndex) ?>
				& <?= row.rValue.toFixed(3) ?>
				\\\hline
			<? } ?>
		\end{tabularx}
	\end{table}
\end{document}